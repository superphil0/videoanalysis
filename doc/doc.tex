\title{Video Analysis\\
1. \"Ubungsaufgabe}
\author{Philipp Omenitsch, xyzxyz\\
Marko Mlinaric, 0825603}
\date{\vspace{-5ex}}

\documentclass[]{scrartcl}

\usepackage{graphicx}
\usepackage{hyperref}

\begin{document}
\maketitle

\section{\"Uberblick zur Implementation und Source-Code}
Die 1. Aufgabe bestand darin ein Verfahren zur Segmentierung von Vordergrundobjekten in Videostreams zu implementieren. Daf\"ur haben wir zuerst den Versuch unternommen, die in der Angabe vorgestellte Methode \textbf{\textit{Color Mean and Variance}} zu implementieren. Als wir allerdings auf Probleme gesto\ss{}en sind, haben wir als Alternative die \textbf{\textit{ViBe}} Methode implementiert \cite{barnich2011vibe}.

Wir haben C++ und OpenCV gew\"ahlt um die Aufgabe zu l\"osen. Unser Source-Code besteht aus den beiden Files \texttt{main.h} und \texttt{main.cpp}.

In Header-File \texttt{main.h} befinden sich Deklarationen von Methoden und globale Variablen, die wir verwendet haben. Der Gro\ss{}teil der Implementation befindet sich im Sourcefile \texttt{main.cpp}. Der Code ist folgendermaßen aufgegliedert:

\paragraph{\texttt{processFrameCMV()}} Unsere Implementierung der \textit{Color Mean and Variance}-Methode. N\"aheres dazu im Abschnitt \ref{sec:cmv}.

\paragraph{\texttt{processFrameVIBE()}} Unsere Implementierung der \textit{ViBe}-Methode. N\"aheres dazu im Abschnitt \ref{sec:vibe}.

\paragraph{\texttt{processFrameMOG()}} Eine Funktion, welche die OpenCV eigene Implementierung des \textit{Gaussian Mixture}-basierten Segmentierungs-Algorithmus' verwendet.

\paragraph{\texttt{processVideo()}} Diese Methode iteriert \"uber die Bild-Dateien und ruft f\"ur jede Datei eine der oben genannten Methoden auf. Sie erh\"alt anschlie\ss{}end das segmentierte Bild und speichert dieses als Datei ab.

\paragraph{\texttt{main()}} Hier werden die Argumente abgearbeitet und die Methode \texttt{processVideo()} aufgerufen.

\section{Color Mean and Variance}\label{sec:cmv}
Schreiben wir hier etwas dazu \dots

\section{ViBe}\label{sec:vibe}
Schreiben wir hier etwas dazu \dots

%\begin{figure}[h!]
%\centering
%\includegraphics[width=\textwidth]{fig.png}
%\caption[]{}
%\label{fig:fix}
%\end{figure}


\bibliographystyle{ieeetr}
\bibliography{main}

\end{document}